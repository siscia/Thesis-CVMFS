%************************************************
\chapter{Implementation}\label{ch:Implementation}
%************************************************

In this chapter we are going to explore how the ingestion inside CVMFS is been
implemented. At first we will describe the write interface of CVMFS, then we
are going to talk how we unpack Docker images to use them with Singularity. We
will move on to describe how we ingest Docker \textit{fat images} while
transforming them into Docker \textit{thin-images} suitable to be used with the
\textit{cvmfs/graphdriver} plugin, moreover we will explore the custom
modification done to CVMFS in order to accommodate the needs of layer
ingestion. We will then explore how all the bookkeeping of images and layers is
manages.  Then we are going to understand how the images are removed from the
file system. Finally we will show the interface given to the administrator.

All the work presented in this chapter is implemented in the
\textit{repository-manager}, a command line utility written in the Go(lang)
language \cite{repository-manager}.

\section{CVMFS Write Interface}

As mentioned in Section \ref{subsec:cvmfs-details} CVMFS is implemented with a
Client-Server architecture: while the client is strictly read-only, the server
does provide a write interface. What we write in the server is what the client
is then able to read.  The write interface of CVMFS is a transactional
interface, hence it is possible to open a transaction, modify the file system and
then either publish the modification or abort the transaction. Those actions
are carried out respectively by the command `cvmfs\_server transaction`,
`cvmfs\_server publish` and `cvmfs\_server abort`.  When a transaction is open
the server file system is a standard writable linux file system, hence it is
possible to modify it using the standard POSIX API, Linux commands or even
graphical file explorers.  Moreover it is possible to test locally the new file
system without actually commit the changes. Of course these actions are not
available to the clients that have access only to a read-only interface. 

Finally it is important to keep in mind that only a single transaction can be
opened at any given time since CVMFS will refuse to open a second transaction. 

\section{Singularity Ingestion}

The step to ingest a singularity images are pretty straightforward. Initially
the image is downloaded from the remote registry and stored in a temporary
area. The download is carried out by Singularity itself, in order to minimize
the possibilities of inconsistencies or of errors. Once the download completed
successfully and the unpacked container file-system is in the local file-system
we start the real ingestion phase.

The first step of the ingestion is to open a transaction in CVMFS. Once we open
the transaction we copy the temporary directory into the CVMFS filesystem under
the “.flat/” directory. Then we commit the first transaction. A second
transaction takes care of creating the symbolic link as described in
\ref{sec:methodology-singularity}.

This few steps are sufficient to make the Singularity images available through
CVMFS. The pseudocode \ref{lst:ingest-singularity} show the details of the
algorithm.

\begin{figure}
\begin{lstlisting}[caption={Algorithm to unpack a Docker image with Singularity and ingest it into CVMFS}, label={lst:ingest-singularity}]
Function UnpackAndIngestDockerImage
        Pass In: DockerImageName
        
        Digest := RetrieveDigestFromImageName DockerImageName
        TemporaryDirectory := CreateTempDirectory
        UnpackDockerImageWithSingularity TemporaryDirectory

        StartCVMFSTransaction
            FlatDir := CreateFlatDirectory Digest
            MoveFrom TemporaryDirectory Into FlatDir
        CommitCVMFSTransaction

        HumanReadableName := GetDirectoryFromDockerImageName DockerImageName
        
        StartCVMFSTransaction
            CreateDirectory HumanReadableName
            CreatSymlinkFrom HumanReadableName To FlatDir
        CommitCVMFSTransaction

EndFunction
\end{lstlisting}
\end{figure}

\section{Docker Ingestion}

Converting Docker \textit{fat images} into Docker \textit{thin images} is a
more complex task than simply make the unpacked image available to use with
Singularity. We will explore all the details of this process in this section.

In the first part we will introduce how CVMFS is able to directly ingest tar
file which is the format used to distribute Docker images as mention in Section
\ref{sec:containers}. Then we will explore how starting from the Docker manifest
we add the layers to the CVMFS file system, then we show how we create the
\textit{thin image} itself.

Moreover we also upload the Docker \textit{thin image} to a Docker registry.

\subsection{CVMFS Ingestion of Tarball}

As mention in \ref{sec:containers} Docker layers are distributed as tar files.
In order to support the use case of ingesting Docker layer we decided to add a
new command to CVMFS `cvmfs\_server ingest`. The ingest command takes as input
a tar file and a directory inside the CVMFS file system and extracts all the
files and directories in the tar file into the directory provided as input. This
command implicitly opens and commits a CVMFS transaction, hence it is possible to
have only a single concurrent ingestion.

\subsection{Docker Ingestion Algorithm}

The first step of the algorithm is to retrieve the manifest of the Docker
images from the Docker registry, as soon as we have the manifest the algorithm
checks if the specific image is already been successfully ingested into the
file system.  This check happens using the metadata stored in the
\textit{.metadata} directory as mention in Section \ref{sec:methodology-keep-track}.
The check consists in a simple comparison between the digest of the manifest of
Docker images just downloaded and the digest of the images already stored in
the \textit{.metadata} folder. If the images is already in the file system the
algorithm terminates.

The next step is to ingest each layer of the Docker image into the
CVMFS file system. As previously we check if the layers already exist in the
file system itself. Since the layers are stored under a path that embeds theirs
own digest as describe in Section \ref{sec:methodology-docker}, checking if a layer is already in the
file system consists in simply checking if the folder where we would ingest the
layer already exists or not.  If the layer already exists we move to the next
layer of the image.

The ingestion of a layer follows a similar procedure of the ingestion of an
unpacked image, the layer is first downloaded into a temporary directory and
then it is ingested using the `cvmfs\_server ingest` command. Another option could
have been to avoid storing the layer in the temporary directory and simply let
the `cvmfs\_serve ingest` command read the content of tar file from
\texttt{STDIN}, we decided against this approach since a non negligible amount
of times the download of the layer fails in the middle wasting all the work
already done by the 'ingest` command.

If the ingestion of any of the layers fails we stop the whole algorithm and we
rely on retries from the administrator in order to have the Docker image served
on CVMFS.

After all the layers have been successfully ingested the next step of the
algorithm is the creation of the Docker \textit{thin image}. The Docker
\textit{thin image} is a standard Docker image which content is a single file
\texttt{thin.json} that contains the set of layers to mount before to start the
container itself as mentioned in Section \ref{subsec:docker-thin-images}. To create this image it is
sufficient to encode the location of the layers into a \texttt{.json} file and then
pack this file into a standard Docker image. The Docker \textit{thin image} is
then uploaded into the Docker registry.

The last step of the algorithm is to store the metadata information about the
image just ingested in order to avoid to repeat work already done. This is done
simply storing the manifest of the docker image in the \textit{.metadata}
folder following the schema presented in Section \ref{sec:methodology-keep-track}.

\section{Garbage Collection of Images}\label{sec:implementation-gc}

We have described how we can add new images to the CVMFS file system, however
updating an image is quite common, especially if the images are referred by
mutable tag such as “latest” which actually represent the latest version of a
particular application.

During the update of an image we avoid to immediately delete the files from the
CVMFS repository, as mentioned in Section \ref{sec:methodology-docker} this could cause
disruption of service for users.

Instead we keep track of all the images that are not necessary anymore in a
specific file, the \texttt{remove-schedule.json} file which is stored in the
hidden directory \textit{.metadata/} just below the main root of the CVMFS file
system. The \texttt{remove-schedule.json} files contains a collections of the
manifest of all the images we are not interested in anymore.

When it is time to actually delete all the old images we scan the
\texttt{remove-schedule.json} file and we carry out the actual removal of the
images. 

The removal of a singularity image is quite simple, indeed, is sufficient to
remove the whole directory.

Removing the layers of the docker images is more complex. At first we need to
identify all the layers that we need to check. This is simple since this
information is stored in the manifest itself which is stored in the
\texttt{remove-schedule.json} file.

Then, for each layer we obtain the list of images that need the layer itself.
From that list we remove the image we are eliminating from the file system. If
the list is now empty we proceed to remove also that specific layer from the
file system.

\section{Administrator Interface}

In order to store images in the CVMFS file system is necessary to know what
image to store, in which repository store it and how to call the respective
thin-image. We decide to call the triplet \texttt{<Input image, CVMFS
Repository, Output image>} a \textit{wish}.

To express a list of those wishes we opted for a simple YAML file that store a
specialization of a generic “wish list”. In the YAML file we specify a list of
Input images, only a single CVMFS repository and a single syntactical
transformation for the Output images. An example of this specific \textit{wish
list} is show on Listing \ref{lst:wish-list}.

\begin{minipage}{\linewidth}
\begin{lstlisting}[caption={Example of a small \textit{wish-list}},label={lst:wish-list}]
version: 1
user: smosciat
cvmfs_repo: 'thin.osg.cern.ch'
output_format: '$(scheme)://gitlab-registry.cern.ch/smosciat/thin-osg/$(image)'
input:
    - 'https://registry.hub.docker.com/library/centos:latest'
    - 'https://registry.hub.docker.com/library/centos:centos6'
    - 'https://registry.hub.docker.com/library/centos:centos7'
    - 'https://registry.hub.docker.com/library/debian:latest'
    - 'https://registry.hub.docker.com/library/debian:stable'
    - 'https://registry.hub.docker.com/library/debian:testing'
    - 'https://registry.hub.docker.com/library/debian:unstable'
    - 'https://registry.hub.docker.com/library/ubuntu:latest'
    - 'https://registry.hub.docker.com/library/fedora:latest'
    - 'https://registry.hub.docker.com/library/python:latest'
    - 'https://registry.hub.docker.com/library/python:2.7'
    - 'https://registry.hub.docker.com/library/python:3.4'
    - 'https://registry.hub.docker.com/library/openjdk:latest'
    - 'https://registry.hub.docker.com/library/openjdk:8'
    - 'https://registry.hub.docker.com/library/openjdk:9'
    - 'https://registry.hub.docker.com/library/gcc:latest'
    - 'https://registry.hub.docker.com/library/r-base:latest'
    - 'https://registry.hub.docker.com/continuumio/anaconda:latest'
    - 'https://registry.hub.docker.com/bbockelm/cms:rhel6'
    - 'https://registry.hub.docker.com/bbockelm/cms:rhel7'
    - 'https://registry.hub.docker.com/efajardo/docker-cms:tensorflow'
    - 'https://registry.hub.docker.com/lincolnbryant/atlas-wn:latest'
\end{lstlisting}
\end{minipage}

The syntactical transformation depends on the input image and is applied to
obtain the final name of the output image. It simply replace the placeholder
\texttt{\$(PLACEHOLDER)} on the \texttt{Output Image}  with respective item of
the \texttt{Input Image}. We show a reference table on Table \ref{table:placeholder}.

\begin{table}
\centering
\begin{tabular}{l|l} 
\toprule
    \texttt{\$(scheme)}     & \texttt{https}                    \\ 
\hline
    \texttt{\$(registry)}   & \texttt{registry.hub.docker.com}  \\ 
\hline
    \texttt{\$(repository)} & \texttt{library/centos}           \\ 
\hline
    \texttt{\$(reference)}  & \texttt{6}                        \\ 
\hline
    \texttt{\$(image)}      & \texttt{library/centos:6}         \\
\bottomrule
\end{tabular}
    \caption{Available placeholders and their application to the image \texttt{https://registry.hub.docker.com/library/centos:centos6}}
\label{table:placeholder}
\end{table}

The use of a simple YAML file to express the desired content of the file system
brings several benefits.  Since the wish list can be hosted on a
version-control system like Github or Gitlab can be used to host and keep track
of the several version of the \textit{wish-list}. Moreover it enables a
Pull-Request based approach to change the wish list itself. Users who wish a
new image to be added to the repository can simply make a pull request adding
their image to the wish list. The administrator of the system act as a
gatekeeper, inspect the image that has been required to be added and decides if it is the case to add
the image or not.

\section{The \textit{repository-manager} command line tool}

All the work presented in this chapter has been implemented in the
\textit{repository-manager} \cite{repository-manager} command line tool. The
tool provide to the administrator of the repository just two main command. The
\textit{convert} command and the \textit{garbage-collect} command.

The \textit{convert} command takes as input a \textit{wish list} as show in
Section \ref{lst:wish-list} and follow all the procedures to add the unpacked Docker
images to be used with Singularity to the repository as well as creating the
several Docker \textit{thin images} and push them into the Docker registries.
Moreover, whenever it needs to delete an image it adds it to the
\texttt{remove-schedule.json} file. 

The actual removing of the images is carried out by the
\textit{garbage-collect} command which reads the \texttt{remove-schedule.json}
files and carried out the removal as described in Section \ref{sec:implementation-gc}.

Finally another command of the utility is worth to mention, is the \textit{loop}
command, which simply executes the \textit{convert} command in an infinite loop, each
time reading again the \textit{wish list} file. This allows the administrator to
run the conversion in the background and set up a periodic job that updates the
\textit{wish list} file downloading it from a version control system.


