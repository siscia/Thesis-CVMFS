%************************************************
\chapter{Conclusion and Future Works}\label{ch:FutureWorks}
%************************************************

In this work we have presented a way to merge efficient run time dependency
management using containers technologies with efficient content distribution
provided by CVMFS. 

The proposed methodology exposed on Chapter \ref{ch:Methodology} is based on
providing a POSIX file system that container runtime can use to load their
content. The use of container technologies allows us to efficiently manage
runtime dependencies logically packing them in different layers exposed as
directories.  We build such file system on CVMFS to exploit its efficient
content distribution mechanism. This forced some of our design decision
regarding the file system tree like the use of \textit{super directories},
however the exact same structure is applicable also to different file system,
not only to CVMFS. 

The proposed methodology have been implemented and explored on Chapter
\ref{ch:Implementation} we have implement a software capable of creating and
managing the whole proposed file system from ingesting the containers content
to deleting it without breaking old images. Moreover we have concluded the work
started with the \textit{cvmfs/graphdriver} that provided only a way to use the
\textit{thin images} while we finally propose a methodology to easily produce
them starting from stander Docker \textit{fat} image and deploy the
\textit{thin images} on standard Docker Registries to easy distribution.

Finally on Chapter \ref{ch:Results} we have show the advantages that the
proposed system bring in terms of startup time of uncached containers and
bandwidth consumption while at the same time not imposing run time penalties.

The final work is in a stable state and there are plans to actually deploy it
in production inside CERN, nevertheless several further advancement are
possible with the respect of increasing the set of supported containers
run-times or improving the software that manages the file system.

For what concerns the run-times we believe that similar work to the one done
for Docker with the \textit{cvmfs/graphdriver} plugin can be done also for
different technologies. In particular we are confident that the approach of
\textit{thin images} can be adopted also by \textit{containerd}
\cite{containerd}, this time implementing a custom \textit{snapshotter} instead
of a Docker graphdriver plugin. Different container run-times necessity further
investigation.

A big limitation to lazily serve the content for containers using lazy systems
like CVMFS is the OCI standard. Container run-times that follow the OCI
standard do not exposes any interface to actually load custom content into the
container itself and relies simply on tarballs to provide the content, hence
\textit{hacks} like the one in the \textit{cvmfs/graphdriver} are necessary. A
solution could be a custom container run-time which provide the possibility to
specify how the content should be loaded and from where.

Another improvement to the system is about the tool that build the file system
structure. It could be re-structured to limit the cyclomatic complexity in
order to allow further enhancement. Another big improvement would be to exploit
to the fullest the transactional interface of CVMFS. Indeed several transaction
are used when ingesting a new image into the file system. An improvement would
be to use only a single transaction for image.

