%************************************************
\chapter{State of the Art}\label{ch:SoA}
%************************************************

Other than the distribution of content, CERN faced the problem of managing
run-time dependencies. A possible solution to this problem would be to
statically link all the dependencies, but this would generate extremely big
binaries.  Another solution would be to carefully managing the all the software
installed on the machine, including all the recursive dependencies. However this
clash with the WLCG model where jobs can migrate from a data center to another
along with their dependencies.  Moreover sometimes even a very careful
installation is not sufficient since is possible that two application that
could run on the same machine have clashing dependencies.

The problem of how to manage runtime dependency is been solved outside CERN in
several different way.

On small scale is possible to use simple package managers that automatically
install the dependencies, however they suffer of few different problem. First
and foremost in case of dependencies clashing the package manager simply refuse
to install the required package, which is not an acceptable solution. Then at
the scale of the WLCG the package manager quickly become a bottleneck with the
respect of the bandwidth necessary to distribute the content but also with the
respect of the size of the internal databases.

Another possible solution is the use of containers. Packaging all the runtime
dependencies along the application code in an immutable file system allow
containers to exactly reproduce the same environment and condition to ensure
smooth operations. The use of containers however is not convenient in very
large clusters like the WLCG, indeed the content necessary to run a container
is distribute as few large tar files and this makes the distribution itself
inefficient since a lot of content not useful for the application itself is
downloaded anyway. 

Several system have been propose to decrease the downloading time of containers
images, however all of them work \textit{ahead of runtime} trying to optimize
for how fast the overall system is capable of delivering the whole image into
the host. FID (Faster Image Distribution) \cite{FID} is a P2P Docker images
distribution system that is able to accelerate the speed of distributing Docker
images by taking full advantage of the bandwidth of not only the Docker
Registry but also of the other nodes in the cluster, while decreasing the
downloading time FID still require to download all the image before to run the
computation. Another work by Anwar et all. \cite{210500} characterize the
workload of large-scale registries in order to derive design implication for
more optimize registries, still they are working to optmize the time necessary
to serve and download the whole image from the registry.

While the aim of this work is still to decrease the start up time of uncached
containers we took a different road. While the previous works focus on optimize
the delivering time of the whole content image we decide to focus on minimize
the amount of content that the client needs to download which of course lead to
shorter start up time for the containers and also on saving bandwidth.

Few other system have been proposed to address the problem using a similar
architecture. The \textit{cvmfs/graphdriver} \cite{graphdriver-plugin} allows
us to run Docker images starting from a \textit{recipe} file that  list the
layers to mount and their location in the file system. Unfortunately a way to
provide the \textit{thin images} was not provided, moreover it wasn't provide a
way to structure the file system itself. Slacker \cite{slacker} tackle the same
problem but it use a very different implementation. Their work is based on NFS
and flatten layers. All the layer of an image get flatten into a single layer,
and such layer stored as a single file into a NFS shared between the Docker
Registries and the Docker Client. While \textit{cvmfs/graphdriver} share a set
of layers as Docker images, Slacker share a simple reference to the file stored
in the NFS. Slacker is able to reach very interesting performance thank to the
implementation of the NFS they use that relies on network disks. We lack the
details about the NFS used by Slacker (Tintri 620) but we believe that their
performance may be strongly influenced by the size of the cluster.

In this work we aim to finally bridge the two world of containers and
distributed read-only file system like CVMFS.


