%************************************************
\chapter{Problem Definition}\label{ch:ProblemDefinition}
%************************************************

In the previous sections we addressed how CERN has overcome the challenge of
software distributions using CVMFS. Unfortunately this is not enough since
run-time dependencies between software components can break application that
instead are running perfectly fine in a different environment.

A possible solution to this problem is the use of container technology.
Packaging an application and all its dependencies in an immutable environment
where it can be fully tested in isolation and then, when we are confident that
the environment capture all the possible run-time dependencies, deploy not the
single application, but the application along with all its environment.

The industry already provide the necessary software for running containers,
however the distribution of the content of the containers itself still relies
on downloading all the possibly necessary file before to running the
computation that introduce a waste of time and network bandwidth. 

In this work we aim to create a read-only file-system structure implementable
in CVMFS that will allow us to bridge together the efficient content
distribution provide by CVMFS with the encapsulation (NON È IL TERMINE
CORRETTO) provide by container technologies.

This work will focus on a specific container technology, Docker. Indeed we
focus on describe a file-system structure suitable to run Docker images using
Singularity and the thin-images Docker graphdriver plugin.

The proposed file-system structure aims to the following goals:
\begin{itemize}
\item Minimize the amount of space required
\item Minimize the start up time of containers
\item Minimize the time necessary to add a new docker image into the file-system
\item Limit the number of files in the CVMFS catalogs 
\item Minimize the complexity of managing the filesystem itself
\end{itemize}

