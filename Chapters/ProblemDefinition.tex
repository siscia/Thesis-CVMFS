%************************************************
\chapter{Problem Definition}\label{ch:ProblemDefinition}
%************************************************

In the Background on chapter \ref{ch:background} we addressed how CERN has
overcome the challenge of software distributions using CVMFS. Unfortunately
this is not enough since run-time dependencies between software components can
break application that instead are running perfectly fine in a different
environment.

A possible solution to this problem is the use of container technology and we
explore how both Singularity and Docker with the \texttt{cvmfs/graphdriver}
plugin are capable to run Docker images hosted in CVMFS avoiding to download
unnecessary from remote host saving bandwidth.

In this work we aim to introduce a read-only file-system structure
implementable in CVMFS that will allow us to bridge together the efficient
content distribution provide by CVMFS with the encapsulation of runtime
dependencies provide by container technologies.

This work will focus on a specific container technology, Docker. Indeed we
focus on describe a file-system structure suitable to run Docker images using
Singularity and the thin-images Docker graphdriver plugin.

The proposed file-system structure aims to the following goals:
\begin{itemize}
\item Minimize the amount of space required
\item Minimize the start up time of containers
\item Minimize the time necessary to add a new docker image into the file-system
\item Minimize the number of files in the CVMFS sub-catalogs 
\item Minimize the complexity of managing the files-ystem itself
\item Provide a great usability for the users.
\end{itemize}

Some of those goal can be measured to asses if we were successful or not, in
particular we are going to:

\begin{itemize}
\item Compare the amount of space required by the CVMFS repository versus the
        amount of space required for each layer.
\item Compare the start-up time of containers hosted in CVMFS versus the
        start-up time of containers not hosted on CVMFS
\item Provide the distribution of number of files in each sub-catalog
\item Measure the complexity of managing the file-system using as a proxy the
        cyclomatic complexity of the software that actually create and manage
                the file-system.
\end{itemize}

Unfortunately we didn't find usable proxy for measuring the usability of the
file-system structure nor it make sense to measure the time necessary to add a
new Docker image into the file-system since it depends on too many variable to
provide a useful measure.
