%*******************************************************
% Abstract
%*******************************************************
%\renewcommand{\abstractname}{Abstract}
\pdfbookmark[1]{Abstract}{Abstract}
% \addcontentsline{toc}{chapter}{\tocEntry{Abstract}}
\begingroup
\let\clearpage\relax
\let\cleardoublepage\relax
\let\cleardoublepage\relax

\chapter*{Abstract}
The problem of software distribution has always been an issue at CERN, given
the massive dimension of the software used for analyzing the data they collect
from the LHC. The problem of software distribution is been solved by the use of
a distributed, read-only file system like CernVM-FS that allows distributing
binaries to all the geographically distributed data-centers used for High
Energy Physics computation. However, the problem of managing run-time
dependencies is still open, indeed, some application can run perfectly well in
an environment while not working in a different environment.

The problem of run-time dependencies is already been solved in the industry
with the use of containers, immutable computing environment which encapsulate
all the run-time dependencies of an application allowing it to run with ease on
different machines. Moreover, containers are becoming always more widespread
also inside CERN, suggesting that they will be a key component for running
future High Energy Physics workload.

However, containers are difficult to distribute in an efficient way since their
content is stored inside a few large files, moreover it has been shown that
most of the content of containers is not used while running the application
itself. Efficient content distribution and managing runtime dependencies should
not be contrasting goals. Hence, in this work, we present a way to efficiently
distribute the content of containers in order to avoid waste of bandwidth and
time.

\newpage

\begin{otherlanguage}{italian}
\chapter*{Sommario}

Il problema della distribuzione del software è sempre stato pressante al CERN,
        questo si spiega visto l'enorme dimensione del software usato per
        analizzare i dati raccolti dal LHC. Il problema della distribuzione del
        software è stato risolto con l'uso di read-only file system distribuiti
        come CernVM-FS che permette la distribuzione dei binari verso tutti i
        data-center geograficamente distribuiti usati nella ricerca della
        fisica delle alte energie. Però, il problema di gestire le dipendenze a
        run-time del software è rimasto intoccato, infatti qualche applicazione
        può funzionare perfettamente in un ambiente mentre non funzionare in un
        ambiente diverso. Il problema della gestione delle dipendeze a run-time
        è stato risolto nell'industria con l'uso dei containers, ambienti di
        computazione immutabili che raccolgono tutte le dipendenze a run-time
        di una applicazione permettendo di far girare la stessa applicazione,
        semplicemente, su macchine diverse. In più containers stanno diventando
        sempre più diffusi anche all'interno del CERN, suggerendo che saranno
        una componente chiave per applicative della fisica delle alte energie
        nel prossimo futuro. Sfortunatamente i containers sono difficili da
        distribuire in modo efficiente dato che sono memorizzati come pochi
        grandi file, inoltre è stato anche mostrato come la magior parte del
        contenuto di un contaner non è realmente usato quando si esegue la
        applicazione stessa. Distribuzione efficiente del contenuto e gestione
        delle dipendenze a run time non dovrebbero essere obbiettivi
        constrastanti. Quindi, in questo lavoro presentiamo un modo per
        distribuire il contenuto di containers in modo efficiente per evitare
        sprechi di banda e tempo.

\end{otherlanguage}

\endgroup

\vfill
